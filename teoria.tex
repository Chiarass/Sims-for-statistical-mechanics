\documentclass{article}
\usepackage[utf8]{inputenc}
\usepackage{graphicx} % Required for inserting images
\usepackage{hyperref}
\usepackage{bbold}
\usepackage{amsmath}
\usepackage{amssymb}
\usepackage{enumerate}
\usepackage{geometry}
\usepackage{inputenc}
\usepackage{mathtools}
\usepackage{listings}
\usepackage{footnote}
\title{Meccanica Statistica}
\author{Chiara Baldelli}
\date{Settembre-Dicembre 2024}

\begin{document}

\maketitle

\section{Introduzione}
La nostra percezione del mondo si ferma al livello macroscopico, eppure siamo consapevoli dell'esistenza di una realtà microscopica composta da atomi e molecole alla base della materia. L'approccio meccanico Newtoniano risulta tuttavia insufficiente per studiare quetso tipo di sistemi pochè in 1 mole di gas è presente un numero di atomi dell'ordine di $10^23$ che rende computazionalmente impossibile scrivere per ognuno degli atomi l'equazione del moto e risolverla risalendo alla dinamica del sistema. Inoltre, ammesso e non concesso che si riesca a fare un'analisi di questo tipo, la dinamica risulta caotica cioè estremamente sensibile alle condizioni iniziali che sono sempre soggette a un errore: anche in questo caso non si riesce a descrivere l'evoluzione di un simile sistema. Un primo approccio allo studio dei gas è stata la termodinamica classica, una teoria fisica che si basa sullo studio di proprietà macroscopiche dei gas (temperatura, pressione, volume) e che riesce a descrivere il loro comportamento attraverso una singola equazione detta "equazione di stato". Nel caso molto particolare di gas perfetto l'equazione di stato assume una forma particolarmente semplice:
\begin{equation*}
    PV=Nk_BT
\end{equation*}
Il grande limite della termodinamica è che essa non spiega davvero cosa succede a livello microscopico, si limita a fornire una descrizione macroscopica dei fenomeni senza entrare nel dettaglio. Per questo motivo nasce l'esigenza di creare una nuova teoria, sviluppata a cavallo tra il 1800 e il 1900 da menti come Einstein, Gibbs e Boltzmann. Questo nuovo approccio, che prenderà il nome di meccanica statistica, si basa su una descrizione statistica e probabilistica del comportamento di un sistema, in linea con le nostre esigenze di conoscere le quantità macroscopiche di esso. Per una trattazione di questo tipo è necessario dare dei fondamenti di teoria della probabilità e delle distribuzioni statistiche, utilizzando un approccio probabilistico assiomatico.
\section{Teoria della probabilità e delle distribuzioni}
Ricordiamo velocemente gli assiomi della probabilità. Sia $P(A)$ la probabilità che un determinato evento si verifichi. Allora essa deve soddisfare 4 assiomi fondamentali, basati su proprietà "ragionevoli":
\begin{enumerate}
    \item $P(A)\geq 0$, cioè la probabilità è sempre un numero positivo
    \item $\sum_i P(A_i)=1$, cioè  se considero tutti i possibili eventi uno di essi deve accadere e quindi la somma delle probabilità è 1
    \item $P(A\vee B)=P(A)+P(B)$, cioè la probabilità che avvengano due eventi indipendenti è la somma delle probabilità dei singoli eventi
    \item La probabilità che si verifichino due eventi dipendenti è il prodotto delle probabilità dei singoli eventi, cioè $P(A \wedge B=P(A)P(B)$
\end{enumerate}
\end{document}
